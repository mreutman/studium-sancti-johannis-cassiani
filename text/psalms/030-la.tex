\psalmChapterWithInscription{30}
{\verseNumInscription{1} In finem psalmus Davídi pro èxtase}
{%%%%%%%%%%%%%%%%%%%%%%%%%%%%%%%%%%%%%%%%%%%%%%%%%%%%%%%%%%%%%%%%%%%%%%%%%%%%%%
\verseNum{2}~\verseFirstLetter{I}n te, Dòmine, sperávi, non confùndar in aetèrnum~; in justìçia tua líbera me. 
\verseNum{3}~Inclína ad me aurem tuam~; acċèlera ut éruas me. Esto mìhi in Deum protectórem, et in domum refúġii, ut sàlvum me fàċias, 
\verseNum{4}~quóniam fortitúdo mea et refúġium meum es tu, et pròpter nomen tuum dedúces me et enútries me. 
\verseNum{5}~Edúces me de làqueo hoc quem absconduérunt mìhi, quóniam tu es protèctor meus. 
\verseNum{6}~In mànüs tuas commèndo spíritum meum~; redemìsti me, Dòmine, Deus veritátis. 
\verseNum{7}~Odìsti observàntes vanitátes supervácue~; ego autem in Dòmino sperávi. 
\verseNum{8}~Exsultábo, et laetábor in mizericórdia tua, quóniam respexìsti humilitátem meam~; salvavìsti de neċessitátibus ànimam meam. 
\verseNum{9}~Nec conclusìsti me in mànibus inimíci~; statuvìsti in loco spaçióso pèdes meos. 
\verseNum{10}~Mizerére mei, Dòmine, quóniam trìbulor~; conturbátus est in ira òculus meus, ànima mea, et vènter meus. 
\verseNum{11}~Quóniam deféċit in dolóre vita mea, et anni mei in ġemìtibus. infirmáta est in paupertáte vìrtüs mea, et òssua mea conturbáta sunt. 
\verseNum{12}~Super òmnes inimícos meos fàctus sum oppróbrium, et viċínïs meïs vàlde, et tìmor notïs meïs. Qui vidébant me foras fuġivérunt a me. 
\verseNum{13}~Oblivióni datus sum, tàmquam mórtuus a corde~; fàctus sum tàmquam vas pèrditum, 
\verseNum{14}~quóniam audívi vituperaçiónem multórum commoràntium in çircùmitu. In eo dum convenírent sìmul advèrsum me, acċìpere ànimam meam consiliáti sunt. 
\verseNum{15}~Ego autem in te sperávi, Dòmine~; dixi, «~Deus meus es tu.~»
\verseNum{16}~In mànibus tuïs sortes meae~; éripe me de mànu inimicórum meórum, et a persequèntibus me. 
\verseNum{17}~Illústra fàċiem tuam super servum tuum~; sàlvum me fac in mizericórdia tua. 
\verseNum{18}~Dòmine, non confùndar, quóniam invocávi te. Erubéscant ìmpii, et deducàntur in infèrnum. 
\verseNum{19}~Muta fiant lábia dolósa, quae loquùntur advèrsus justum iniquitátem in supèrbia et in abuzióne. 
\verseNum{20}~Quam magna multitúdo dulċédinis tuae, Dòmine, quam absconduìsti timèntibus te~; perfeċìsti eïs qui sperant in te, in conspèctu filiórum hòminum. 
\verseNum{21}~Abscòndes eos in abscòndito faċiéi tuae a conturbaçióne hòminum~; próteges eos in tabernáculo tuo a contradicçióne lingüárum. 
\verseNum{22}~Benedìctus Dòminus, quóniam mirificávit mizericórdiam suam mìhi in çivitáte muníta. 
\verseNum{23}~Ego autem dixi in exċèssu mentis meae, «~Projèctus sum a fàċie oculórum tuórum.~» Ìdeo exaudivìsti voċem oraçiónis meae, dum clamárem ad te. 
\verseNum{24}~Dilìgite Dòminum, òmnes sancti ejus, quóniam veritátem requíret Dòminus, et retrìbuet abundànter faċièntibus supèrbiam. 
\verseNum{25}~Viríliter ágite, et confortétur cor vèstrum, òmnes qui sperátis in Dòmino. 
}
