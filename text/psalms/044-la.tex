\psalmChapterWithInscription{44}
{\verseNumInscription{1} In finem pro eïs qui commutabùntur fíliïs Córachi ad intelléctum cànticum pro dilécto}
{%%%%%%%%%%%%%%%%%%%%%%%%%%%%%%%%%%%%%%%%%%%%%%%%%%%%%%%%%%%%%%%%%%%%%%%%%%%%%%
\verseNum{2}~\verseFirstLetter{E}ructávit cor meum vèrbum bonum~; dico ego ópera mea reġi. lìngüa mea calamus scribae velóċiter scribèntis. 
\verseNum{3}~Speçiósus forma prae fíliïs hòminum, diffúsa est gráçia in lábiïs tuïs~; proptèrea benedíxit te Deus in aetèrnum. 
\verseNum{4}~Acċìngere glàdio tuo super femur tuum, potentìssime. 
\verseNum{5}~Spèċie tua et pulchritúdine tua intènde, próspere proċéde, et regna, pròpter veritátem et mansuetúdinem et justìçiam, et dedúcet te mirabìliter dèxtera tua. 
\verseNum{6}~Saġìttae tuae acútae~; pòpuli sub te cadent, in corda inimicórum reġis. 
\verseNum{7}~Sedes tua, Deus, in sáeculum sáeculi~; vìrga direcçiónis vìrga regni tui. 
\verseNum{8}~Dilexìsti justìçiam, et odìsti iniquitátem~; proptèrea unxit te Deus, Deus tuus óleo laetìçiae prae consórtibus tuïs. 
\verseNum{9}~Myrra et gutta et càssia a vestimèntïs tuïs, a dómibus evórneïs, e quìbus delectavérunt te~; 
\verseNum{10}~fíliae reġum in honóre tuo~; àstitit reġína a dèxtrïs tuïs in vestítu deauráto, çircùmdata varietáte. 
\verseNum{11}~Audi, fília, et vìde, et inclína aurem tuam, et oblivísċere pòpulum tuum et domum pàtris tui. 
\verseNum{12}~Et concupísċet rex decórem tuum, quóniam ìpse est Dòminus Deus tuus, et adorábunt eum. 
\verseNum{13}~Et fíliae Tyrii in munèribus vùltum tuum deprecabùntur~; òmnes dívites plebis. 
\verseNum{14}~Òmnis glória ejus fíliae reġis intrínsecus, in fìmbriïs áureïs, çircumamìcta varietátibus. 
\verseNum{15}~Adducèntur reġi vìrġines post eam~; pròximae ejus afferèntur tìbi. 
\verseNum{16}~Afferèntur in laetìçia et exsultaçióne~; adducèntur in templum reġis. 
\verseNum{17}~Pro pàtribus tuïs nati sunt tìbi fílii~; constìtues eos prínċipes super òmnem terram. 
\verseNum{18}~Mèmores èrunt nóminis tui in òmni ġeneraçióne et ġeneraçiónem. Proptèrea pòpuli confitebùntur tìbi in aetèrnum, et in sáeculum sáeculi. 
}
